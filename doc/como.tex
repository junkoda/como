\documentclass[a4paper,11pt]{article}
\usepackage{amsmath}
\usepackage{amssymb}
\usepackage{natbib}
\setcitestyle{authoryear, open={(},close={)}}
% Use Times fonts
\usepackage{astro_journal_names}
\usepackage{mathptmx}
\usepackage[scaled]{helvet}
\renewcommand{\ttdefault}{pcr}
\usepackage{bm}

\setlength{\oddsidemargin}{0pt}   %%% left margin
\setlength{\textwidth}{159.2mm}
\setlength{\topmargin}{0mm}
\setlength{\headheight}{10mm}
\setlength{\headsep}{10mm}         %%% length between header and test
\setlength{\textheight}{219.2mm}

%\setlength{\parindent}{0pt}


%\bibliographystyle{abbrvnat}

%\renewcommand{\headrulewidth}{0pt}

\title{COLA mocks for VIPERS PDR2}
\author{Jun Koda}
\date{}

\begin{document}

\maketitle

\section{Overview}

We create 1020 realasations of mock galaxy catalogues for the VIMOS
Public Extragalactic Redshift Survey (VIPERS) Public Data Realase 2
(PDR2) using an approximate simulation method called the COmoving
Lagrangian Acceleration (COLA) simulation
\citep{2013JCAP...06..036T}. We populate the dark matter halo using
the Halo Occupation Distribution (HOD) that matches the VIPERS galaxy
number density and projected correlation function
(\S~\ref{section:hod}). The purpose of these mocks is to compute the
covariace matrix, which requires many realisation.

\begin{table}[h]
\begin{center}
\begin{tabular}{ccc}
  \hline
  region & filename & index $n$\\
  \hline
  W1 & w1/mock\_w1\_n.txt & 00001 -- 01020\\
  W4 & w4/mock\_w4\_n.txt & 00001 -- 01020\\
  \hline
\end{tabular}
\caption{File name convension}
\label{table:filename}
\end{center}
\end{table}


\noindent VIPERS consists of two regions in the sky, W1 and W4; each
mock catalogue is labelled by the region name, w1 or w4, and the index
$n = 1,2,\dots,1020$ (Table~\ref{table:filename}). All mocks are
basically independent --- there are some long-wavelength correlation
when they are created from a common simulation box, but all mocks are
created either from non-overlapping volumes or from independent
simulations. We create 3 mocks for W1 ($n = 3i - 2, 3i - 1$, and $3i$
for $i=1,2,\dots, 340$) or 6 mocks for W4 ($n = 6i-5, \dots, 6i$ for
$i = 1,2,\dots,170$) from one simulation box.

The mock catalogue is an ascii file separated by spaces as descibed in
Table~\ref{table:file-format}. The origin of the comoving coordinate
$\bm{x} = \bm{0}$ is the observer, and the $x_1$ axis is passing the
centre of the survey colume (Fig~??). The `cosmological` redshift
$z_\mathrm{cosmo}$ corresponds to the real space distance from the observer,
$z_\mathrm{RSD}$ includes the Doppler shift due to the peculiar
velocity, and $z_\mathrm{obs}$ includes the redshift error (ref
section ??) in addition.


\begin{table}
\begin{center}
  \begin{tabular}[b]{cccl}
    \hline
    Column &  & Unit & Description\\
    \hline
    1      & ID            & & unique nonnegative integer\\
    2 -- 4 & $x_1,x_2, x_3$ & $[h^{-1} \,\mathrm{Mpc}]$ &
             comoving corrdinate in redshift space\\
    5      & $v_r$ & $[\textrm{km s}^{-1}]$ & line-of-sight peculiar velocity
             including $v_{r,sat}$\\
    6      & $z_\mathrm{cosmo}$ & &
             cosmological redshift; a function of realspace position\\
    7      & $z_\mathrm{RSD}$ & & $z_\mathrm{cosmo} + $
             peculiar velocity Doppler shift\\
    8      & $z_\mathrm{obs}$ & & $z_\mathrm{RSD} + $ measurement error\\
    9      & $M_{200,m}$ & $[h^{-1} M_\odot]$ & host halo mass\\
    10     & $r_{sat}$   & $[h^{-1} \,\mathrm{Mpc}]$ &
             galaxy displacement from halo centre\\
    11     & $v_{r, sat}$ & $[\textrm{km s}^{-1}]$ &
             line-of-sight virial velocity\\
    12     & $w_\mathrm{TSR}$ & & TSR weight\\
    13     & $w_\mathrm{SSR}$ & & SSR weight\\
    14     & quadrant & & name of the quadrant\\
    \hline
  \end{tabular}
  \caption{File format of mock galaxy catalogues}
  \label{table:file-format}
\end{center}
\end{table}

\section{Simulation}

We use a parellel COLA simulation code \citep{2016MNRAS.459.2118K}.
with a cubic periodic box with $600h^{-1} \,\mathrm{Mpc}$ on a side,
$2048^3$ particles and the same number of grids for Particle Mesh
force computation. The cosmological parameters are based on the Big
MultiDark Planck
simulation\footnote{https://www.cosmosim.org/cms/simulations/bigmdpl/},
$\Omega_m = 0.31, \Omega_\Lambda = 0.69, \Omega_b = 0.048, h = 0.67,
\sigma_8 = 0.82, n_s = 0.96$

We setup the initial condition at scale factor $a=0.1$ using the 2nd-order
Lagrangial Pertubation Theory and evolve with timestep $\Delta a =
0.05$. We identify dark matter haloes using the Friends-of-Friends algorithm
\citep{1985ApJ...292..371D} with linking length $\ell = 0.168$ times
the mean particle separation length. We calibrate the halo mass according to the \citet{2010ApJ...724..878T} mass function by matching the halo number density above the halo mass:
\begin{equation}
  n_\mathrm{FoF}(>M_\mathrm{FoF}) = n_\mathrm{Tinker}(>M_{200,m})
\end{equation}
The halo mass $M_{200,m}$ is defined by the mass enclosed by a sphere, within
which the mean density is 200 times the mean matter density.

\section{Lightcone}
We first remap the cubic periodic box to a longer cuboid that can cover
the survey volume \citep{2010ApJS..190..311C}. The remapping is
characterised by three integer vectors $\bm{u}_i$, which specifies
the basis vectors $\bm{e}_i$ as follows:
%
\begin{align}
  \label{eq:remapping}
  \bm{e}_1 &= \bm{u}_1/|\bm{u}_1| \nonumber\\
  \bm{e}_2 &= \bm{u}'_2/|\bm{u}'_2|,
    \quad u'_2  \equiv \bm{u}_2 -
    (\bm{u}_1 \cdot \bm{u}_2/|\bm{u}_1|^2) \bm{u}_1,   \nonumber \\
  \bm{e}_3 &= \bm{e}_1 \times \bm{e}_2.
\end{align}
%

\begin{table}
\begin{center}
\begin{tabular}{ccccc}
\hline
region & remapped boxsize & $\bm{u}_1$ & $\bm{u}_2$ & $\bm{u}_3$\\  
\hline
W1 & ?? & $(2, 2, 1)$ & $(1, 0, 0)$ & $(0, 1, 0)$\\
W4 & ?? & $(2, 2, 1)$ & $(1, 0, 3)$ & $(1, 1, 0)$\\
\hline
\end{tabular}
\end{center}
\caption{Box remapping}
\label{table:box-remapping}
\end{table}

We use 6 snapshots from scale factor $a=0.45$ to $0.70$, separated by
$\Delta a = 0.05$ to cover the redshift range $z= 0.4 -- 1.2$ ($a =
0.455 -- 0.714$). We create a lightcone by extracting haloes in a
distance range that corresponds to $a_i - 0.025 < a < a_i + 0.025$ from a
snapshot at scale factor $a_i$.

\section{Halo Occupation Distribution (HOD)}
\label{section:hod}

We populate dark matter haloes in the COLA simulation with the Halo
Occupation Distribution to match the observed galaxy number and
clustering properties. We use the HOD parameterisation by
\citet{2005ApJ...633..791Z}: The probability that a halo with mass $M$ has
a central galaxy is given by,
%
\begin{equation}
  \langle n_\mathrm{cen} \rangle = \frac{1}{2}
  \left[ 1 + \mathrm{erf}\left( \frac{\log_{10} M - \log_{10} M_\mathrm{min}}{\sigma_{\log M}} \right) \right],
\end{equation}
%
and a mean number of satellite galaxy, under the condition that the halo has a central galaxy, is,
%
\begin{equation}
  \langle n_\mathrm{sat} \rangle = \left( \frac{M - M_0}{M_1} \right)^\alpha.
\end{equation}
%
The distribution contains 5 free parameters, but we fix one parameter,
\begin{equation}
  M_0 = M_\mathrm{min},
\end{equation}
since it does not strongly affect the galaxy statistics
\citep[e.g.][]{2013A&A...557A..54D}.


\begin{align}
  \log_{10} M_\mathrm{min}  &= c_0 + c_1 (z - z_0) + c_2 (z - z_0)^2 + c_3 (z - z_0)^3\\
  M_0                     &= M_\mathrm{min}\\
  \sigma_{\log_{10} M}      &= c_4 + c_5 (z - z_0)\\
  \log_{10} (M_1/M_\mathrm{min}) &= c_6 + c_7 (z - z_0) \\
  \alpha                  &= c_8 + c_9 (z - z_0)
\end{align}

The constant $z_0$ can be set to any value without loss of generatity;

\section{Parent catalogue}

We create mock catalogues in two steps; we first create a
\textit{parent catalogue} which corresponds to a photometric source
catalgue, and then add effects of target sampling
(\S~\ref{section:tsr}), spectroscopic succsess rate
(\S~\ref{section:ssr}), and redshift error (\S~\ref{section:zerror}).



\section{Target Sampling Rate (TSR)}
\label{section:tsr}

\section{Sepectroscopic Success Rate (SSR)}
\label{section:ssr}

\section{Redshift error}
\label{section:zerror}



\bibliographystyle{plainnat}
\bibliography{como}

\label{LastPage}
\end{document}

